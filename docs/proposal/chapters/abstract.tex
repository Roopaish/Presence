The "Presence" project is an innovative approach to attendance tracking in college classrooms. With the help of automation and data analysis, the system aims to provide a more efficient and accurate way of keeping track of student attendance while also providing valuable insights into student behavior and academic performance. The system uses cameras to detect when students enter and leave the classroom. The data collected is then analyzed to provide a real-time attendance report that instructors can access from their devices. This eliminates the need for manual attendance-taking and reduces the potential for errors or inaccuracies. In addition to real-time attendance tracking, the system also provides detailed reports on student attendance patterns, subject preferences, and behavior in class. By analyzing this data, instructors and administrators can identify students who may be struggling with attendance or academic performance and take proactive steps to address these issues. They can also identify trends and patterns in attendance that can inform decisions about course scheduling and curriculum design.\\

The "Presence" project is a valuable tool for both students and faculty. For students, it provides a streamlined attendance tracking process that eliminates the need for manual sign-ins and helps ensure that they are meeting attendance requirements. For faculty, it provides real-time insights into student attendance and behavior that can inform teaching strategies and improve academic outcomes.\\ 

\textbf{Keywords}--
\textit{ automated attendance system, data analysis, college classrooms, attendance tracking, attendance report, academic performance, attendance patterns, subject preferences, student behavior, teaching strategies, curriculum design }