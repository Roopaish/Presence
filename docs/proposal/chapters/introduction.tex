In today's technology-driven era, the trend is to switch from traditional systems to fast, smart, and interactive systems that can be linked to web applications, enabling users to access the system from anywhere at any time. In academic settings, regular attendance is crucial for students to acquire knowledge, participate in class discussions, and engage with their peers. However, traditional methods of taking attendance, such as calling out names or manually signing in, can be tedious, prone to errors, and inefficient.\\

To address these challenges, we propose the design and implementation of a face detection and recognition system called "Presence." The "Presence" system is an automated attendance system that utilizes facial recognition technology to identify and mark the attendance of students attending a lecture in a classroom. By automating the attendance process, "Presence" aims to save time for instructors, reduce errors, and provide a more accurate and efficient way to monitor student attendance.\\

The "Presence" system is an advanced and sophisticated system that uses a camera to detect the faces of students entering and leaving the classroom. The system will be designed to recognize individual faces, match them with the database of registered students, and mark their attendance automatically. The data collected by the system will be stored securely and can be accessed through a web application by instructors, students, and authorized personnel.\\

In addition to recording attendance, the "Presence" system will provide valuable insights into attendance patterns, such as how often a student attends a particular class or how long they stay in the classroom. This data can be analyzed to identify trends and patterns in student behavior, which can be used to make informed decisions about how to improve learning outcomes.